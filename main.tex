%% LyX 2.0.7 created this file.  For more info, see http://www.lyx.org/.
%% Do not edit unless you really know what you are doing.
\documentclass[11pt,twocolumn]{article}
\usepackage[utf8]{inputenc}
\usepackage{graphicx}

\makeatletter
%%%%%%%%%%%%%%%%%%%%%%%%%%%%%% User specified LaTeX commands.




\title{Smarter Merge Tool for Structured Java Code }
\author{Kartik Andalam, Raoul D'Cunha, Shreya Shah }
\date{March 2014}



\makeatother

\begin{document}
\maketitle

%\twocolumn



\section{Abstract}

Software merging is a necessity for large-scale software development
and while there are many code version systems to help with this, none
of them take in account the structure of the code. Understanding semantic
and syntactic structure of the code helps automate merging further,
which in turn requires less effort from the developer at commit.

We propose a tool that would attempt to automate merging and assist
the user in identifying the changes by analysing the semantics of
the code. We intend to use PDStore to store the code in a structured
manner and that can be used in a versioning system. The proposal also
highlights the requirements, milestones and possible evaluation method
of for the research project.


\section{Motivation}

Today's code version systems are based on textual comparisons. Being
textual based means that they do not understand the structure of the
code, and as a result the tools are unable to assist the developer
with code merging. This gave motivation to utilise the extra information
that can aid in the merging .

We aim to develop a language-aware code versioning system based on
PDStore. In particular, we will utilize structured data to visualize
changes to the structure of a source code file to assist with merging.


\section{Problem Statement}

Today's code version systems are based on textual comparisons. Being
textual based means that they do not understand the structure of the
code, and as a result the tools are unable to assist the developer
with code merging. We aim to develop a language-aware code versioning
system based on PDStore. In particular, we will utilize structured
data to visualize changes to the structure of a source code file to
assist with merging.


\section{Literature review}

In \cite{XMLMerge} describes technique for performing a 3 way structured
XML merge. The technique involves defining merge rules derived from
common use cases of XML editing. While there are similar problems,
a few difference exist in the problem. Due to the nature of XML, the
local structure is also to be considered. However with Java code the
structure can vary and still present the same semantic definition.
Thus some of the rules made in \cite{XMLMerge} become invalid for
semistructured data such as code.

\cite{SemanticDiff} describes a tool called “Semantic diff” and techniques
used in that tool to show the effect of modifications. The tool aims
at providing the user with a summarised report of semantic changes
between two version of a procedure, and focuses on minimising any
spurious differences such as, renaming local variables. While the
tool provides knowledge of variable dependences within a procedure,
it is not as helpful when considering an entire program as the tool
does not maintain any relationship between method invocation and method
definition.

\cite{mergegraph} describes the problems associated with common textual
based merging. It proposes a solution that can be applied to software
artifacts of different types. The solution involves representing information
with extra information such as namespaces appended with UIDs. Once
the data is represented in this special structure, the job of analysing
two structures should be made easier.

\cite{CodeClones} proposes a novel method for merging code clones
and presents a successful case study conducted by using the Aries
tool which uses to identify code clones and merge them. The method
described in \cite{CodeClones} consists of a two phase process, where
code clones that can be refactored are quickly detected, and the metrics
are measured which indicate how the detected code clones should be
merged.


\section{Requirements}

A successful execution of the research endeavour should meet the following
requirements: 
\begin{enumerate}
\item To be able to successfully identify simple changes in method body
but not code structure 
\item To be able to successfully identify and common sections of code in
two structures with slight variations 
\item Identify changes in structure, such as method location being shifted 
\item Successfully merge two differing versions of code following from the
examples in 1 and 2
\end{enumerate}
The following optional requirements would also be attempted to met
during the execution of this research endeavour 
\begin{enumerate}
\item To integrate this additional functionality into the existing PDStore
versioning application. 
\end{enumerate}

\section{Design and Implementation}

The underlying technology used in the proposed method involves using
PDStore to be able to form a structured view of the code. The logic
for doing the detection and merging will be written in Scala. After
defining a method to define “sameness” of a section of the code, we
will be using a similar approach to \cite{XMLMerge} to generate rules
that help define merging.

The following shows an example of where our tool successfully merges
two documents, identifying the changes with respect to semantic meaning
and structure.

\begin{figure}
\includegraphics[scale=0.8]{IdentificationOfChanges}\caption{Identification of changes and merged result}
\end{figure}


Figure 1 shows one possible scenario we intend to solve with our tool.
We aim to recognise if a method is carrying out the same procedure
as an existing method, at commit. As you can see in the above figure,
the tool merges the two methods into one, by keeping the recent name
of the method and same code within the method.


\section{Evaluation}

Paper prototypes will be used to evaluate the feasibility of the design
of the User Interface of our tool. This study will reveal both fundamental
and minor flaws in the UI that would need to be fixed before the final
milestone. Specific scenarios will be constructed outlining various
use cases that will then be run through by the participants on multiple
different UI designs. The participants will be asked to think out
loud, so that their thought process can be observed and evaluated.
Feedback will then be collected from the participants at the end through
the use of a questionnaire which will then be used to measure the
usefulness of each design.

Any identified flaws will then be corrected and the test will then
be undertaken again so as to verify that the flaws are indeed fixed.


\section{Project Plan}

Listed below are three milestones for this project %\newline adding a new line puts too much space over here

\begin{description}
\item [{Milestone One:}] Successfully identify the type of simple structural
changes


This involves defining rules and defining a notion of “sameness” for
two artifacts of code. The PDStore database will be used to compare
the two different versions of the code. Completion criteria involves
being able to generate output that clearly identify the changes. \\
\textbf{Due Date:} 15th April 2014

\item [{Milestone Two:}] Perform merging of simple structural changes


After being able to identify the changes, we need to define appropriate
rules for merging.To complete the milestone, the merger will be required
to successfully merge files for a few defined scenarios. \\
\textbf{Due Date:} 30th April 2014

\item [{Milestone Three:}]~


Integrate the added functionality to existing PDStore based versioning
tool. Integrating the added functionality will allow us to conduct
our evaluation on users. This will be key in determining the usefulness
and effectiveness of the changes.\\
\textbf{Due Date:} 12th May 2014 

\end{description}
 \bibliographystyle{plain}
\bibliography{myBib}

\end{document}
