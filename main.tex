%% LyX 2.0.7 created this file.  For more info, see http://www.lyx.org/.
%% Do not edit unless you really know what you are doing.
\documentclass[11pt,twocolumn]{article}
\usepackage[utf8]{inputenc}
\usepackage{graphicx}

\makeatletter
%%%%%%%%%%%%%%%%%%%%%%%%%%%%%% User specified LaTeX commands.








\title{Smarter Merge Tool for Structured Java Code }
\author{Kartik Andalam, Raoul D'Cunha, Shreya Shah }
\date{March 2014}









\makeatother

\begin{document}
\maketitle

%\twocolumn



\section{Abstract}

Software merging is a necessity for large-scale software development
and while there are many code version systems to help with this, many
of them don't take the structure of the code into account. We propose
to develop a merge tool that will utilise the PDStore%
\footnote{PDStore is a Triplestore database developed at the University of Auckland%
} database to maintain and analyse the structure of the data.

The aim of the tool is to utilise information related to the structure
of a file stored in PDStore and highlight the conflicts that arise
in merging. The proposal highlights the requirements, milestones and
possible evaluation methods for the research project.


\section{Motivation}

%Today's code version systems are based on textual comparisons. Being
%textual based means that they do not understand the structure of the
%code, and as a result the tools are unable to fully assist the developer
%with code merging. This gave a reason to utilise the extra information
%that can aid in the merging.


Developing a merge tool that maintains information in a structured
form, gives scope to improve the way that conflicts and changes are
showed. This motivated us to explore the effectiveness and usability
of such merge tools and how they compare to existing merge tools.


\section{Related Work}

\cite{CodeClones} proposes a novel method for merging code clones
and presents a successful case study conducted by using the Aries
tool which uses to identify code clones and merge them. The method
described in \cite{CodeClones} consists of a two phase process, where
code clones that can be refactored are quickly detected, and the metrics
are measured which indicate how the detected code clones should be
merged.

\cite{SemanticDiff} describes a tool called “Semantic diff” and techniques
used in that tool to show the effect of modifications. The tool aims
at providing the user with a summarised report of semantic changes
between two versions of a procedure, and focuses on minimising any
spurious differences such as, renaming local variables. While the
tool provides knowledge of variable dependences within a procedure,
it is not very helpful when considering an entire program because
the tool does not maintain any relationship between method invocation
and method definition.

In \cite{XMLMerge} a technique for performing a 3 way structured
XML merge is described. The technique involves defining merge rules
derived from common use cases of XML editing. While there are similar
problems, a few differences exist. By the nature of XML, the local
structure is important. However with Java code the structure can vary
and still present the same semantic definition. Thus some of the rules
made in \cite{XMLMerge} become invalid for flexible structured data
such as code.

\cite{mergegraph} describes the problems associated with common textual
based merging. It proposes a solution that can be applied to software
artefacts of different types. The solution involves representing information
with extra information such as namespaces appended with UIDs. Once
the data is represented in this special structure, the job of analysing
two structures should be made easier.


\section{Requirements}

A successful execution of the research endeavour should meet the following
requirements: 
\begin{enumerate}
\item To be able to successfully identify simple changes in method body
but not necessarily code structure 
\item To be able to successfully identify and common sections of code in
two structures with slight variations 
\item Identify changes in structure, such as method location being shifted 
\item Successfully merge two differing versions of code following from the
examples in 1 and 2 
\end{enumerate}
The following optional requirements would also be attempted to met
during the execution of this research endeavour 
\begin{enumerate}
\item To integrate this additional functionality into the existing PDStore
versioning application. 
\end{enumerate}

\section{Design and Implementation}

The underlying technology used in the proposed method involves using
PDStore to be able to form a structured view of the code. The merging
logic will be written in Scala, while the GUI will be designed in
Swing.

Figure 1 is an example that illustrates a possible GUI for three way
merge involving PD Tree Views. The title string belonging to the LOTR
book in the My book library database has differing values in each
of the three existing versions(Base, Version A and Version B). We
hope to colour code information such as modified items and or items
in conflict to help better inform the user of the types of changes.
The user can select items from different versions to be included in
the merged representation by clicking on the checkbox beside the items.

\begin{figure}
\centering{}\includegraphics[scale=0.44]{\string"Diagram v2\string".png}\caption{Identification of changes and merged result}
\end{figure}



\section{Evaluation}

This evaluation will consist of an usability testing and comparative
analysis with existing merge tools.

As part of the usability testing, paper prototypes will be used to
evaluate the feasibility of the design of the User Interface of our
tool. This study will reveal both fundamental and minor flaws in the
UI that would need to be fixed before the final milestone. Specific
scenarios will be constructed outlining various use cases that will
then be run through by the participants on multiple different UI designs.
The participants will be asked to think out loud, so that their thought
process can be observed and evaluated. Feedback will then be collected
from the participants at the end through the use of a questionnaire
which will then be used to measure the usefulness of each design.

Any identified flaws will then be corrected and the test will then
be undertaken again so as to verify that the flaws are indeed fixed.

The comparative analysis will consist of test cases outlining major
scenarios will be laid out at the start of the project and will later
be run against existing merge tools and our developed merge tool.
The outcome of these test scenarios will help determine the effectiveness
of the tools under different scenarios


\section{Project Plan}

Listed below are the three milestones for this project %\newline adding a new line puts too much space over here

\begin{description}
\item [{Milestone One:}] Successfully identify the type of simple structural
changes


This involves defining rules and defining a notion of “sameness” for
two code artefacts. The PDStore database will be used to compare the
two different versions of the code. Completion criteria involves being
able to generate output that clearly identify the changes. \\
 \textbf{Due Date:} 15th April 2014

\item [{Milestone Two:}] Perform merging of simple structural changes


After being able to identify the changes, we need to define appropriate
rules for merging. To complete the milestone, the merger will be required
to successfully merge files for a few defined scenarios. \\
 \textbf{Due Date:} 30th April 2014

\item [{Milestone Three:}] Integrate the added functionality to existing
PDStore based versioning tool~


Integrating the added functionality will allow us to conduct our evaluation
on users. This will be key in determining the usefulness and effectiveness
of the changes.\\
 \textbf{Due Date:} 12th May 2014

\end{description}
 \bibliographystyle{plain}
\bibliography{myBib}

\end{document}
